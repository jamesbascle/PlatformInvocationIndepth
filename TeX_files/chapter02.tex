\chapter{Introduction to P/Invoke}

\section{Getting Started}
First, create a Directory as 'ChapterTwo' for this project and create a new file, 'ChapTwo.c' under 'ChapterTwo' folder.

Let's assume we have a basic Addition function that we want to call from C Library.

\begin{lstlisting}{c}
int Sum(int a, int b)
{
  return a+b;
}
\end{lstlisting}

It's a simple Addition Operation at a first glance, but there are\newline considerations that must be observed first before attempting to write platform invocation wrapper code for the function above:

\begin{enumerate}
	\item `int` datatype in C can be considered 2 bytes long or 4 bytes long or however long it may be depending on the architecture and compiler that the library is compiled on. In C Standard, int must be capable of containing \textbf{at least} the [\textminus32,767, +32,767] range; thus, it is at least 16 bits in size.
	
	\item Because of the fact mentioned above, you can reasonably safeguard against data loss by supplementing C\# int integer which contains 4 bytes integer or you may choose follow the standard strictly by supplying C\# short integer which holds 2 bytes integer even though it can suffer data loss. The best approach is to avoid using the "at least" integers in C and use fixed size integers provided by compiler in ''stdint.h'' header if you have Foreign Function Interface kept in mind.
	
	\item Sometime you have to keep Endianess in mind although it is less of a concern in x86\_64 architecture since the little endianess is the default.
\end{enumerate}
\newpage
The best approach to writing Addition function is to make it clear what sized integer you're attempting to add with if possible.

\begin{lstlisting}{c}
#include <stdint.h>
int32_t Sum(int32_t a, int32_t b)
{
  return a+b;
}
\end{lstlisting}

\section{Compiling the Library}
The book assumed you have sufficient knowledge of C, we will still however provide compilation instruction. The following command assumes that you have named your source code file as 'ChapTwo.c'.

\begin{lstlisting}
clang -std=c99 -shared -fPIC -olibChapTwo.so ChapTwo.c
\end{lstlisting}

There are few things that are being done with compiler:

\begin{enumerate}
	\item '-std=c99' specify that we are compiling C source code under C99 Standard.
	\item '-shared' specify that we want the program to be compiled as shared/dynamic library.
	\item '-fPIC' specify that code must be position independent so that library can be loaded in other process and have code be made available to be run anywhere in program address space regardless of code's address.
	\item '-olibChapTwo.so' specify what the output library should be named as. lib prefix in 'libChapTwo.so' is a matter of naming convention to be followed on Linux although compilers like clang and gcc does search libraries based on lib prefix when using '-l' option. 
\end{enumerate}
\newpage
\section{Configuring C\# Project}
Since we're already in "ChapterTwo" directory, we can go ahead and run 'dotnet new Console'. There are a few steps we need to add to our C\# project first, we need to automate compilation process of our C file and to copy the compiled C library to the target directory for either Debug, Release, or any other configurations.

Open up 'ChapterTwo.csproj' file with your favorite editor, and add the following under '</PropertyGroup>' inside '<Project>' tags.

\begin{lstlisting}{language=XML}
<Target Name="CompileCProject" AfterTargets="AfterBuild">
	<exec Command=
	"clang -std=c99 -shared -fPIC -olibChapTwo.so
	ChapTwo.c" />
	<Copy SourceFiles="libChapTwo.so" DestinationFolder="$(OutDir)" />
</Target>
\end{lstlisting}

The snippet above does few things after building our C\# project:
\begin{enumerate}
	\item Compile ChapTwo.c code as a shared library, libChapTwo.so
	\item Copy libChapTwo.so into any target directory that C\# is being built in.
\end{enumerate}

This make it significantly easier to modify our code without having to run any additional commands for it to take effect.

Your CSProj should look like the following:

\begin{lstlisting}{language=XML}
<Project Sdk="Microsoft.NET.Sdk">

 <PropertyGroup>
  <OutputType>Exe</OutputType>
  <TargetFramework>netcoreapp2.0</TargetFramework>
 </PropertyGroup>
 <Target Name="CompileCProject" AfterTargets="AfterBuild">
  <exec Command="clang -std=c99 -shared -fPIC -olibChapTwo.so
ChapTwo.c" />
  <Copy SourceFiles="libChapTwo.so" DestinationFolder="$(OutDir)" />
 </Target>
</Project>
\end{lstlisting}
\newpage
\section{Wrapping C Code in C\#}
Open up Program.cs, add a new using directive at the top of your source code.

\begin{lstlisting}{c}
using System.Runtime.InteropServices;
\end{lstlisting}

This line imports all of the platform invocation services which enables us to interacts with C library with ease.

Add the following lines under Program class:

\begin{lstlisting}{c}
[DllImport("ChapTwo")]
static extern int Add(int a, int b);
\end{lstlisting}

The DllImport attribute declare that static externally defined function is defined in C library and to have CLR create Platform Invocation stub to define the said function with external library.

It is required to declare the function with static and extern modifiers since it is a function that is independent of state and is externally defined.

Finally, modify the ''Console.WriteLine'' line to the following:

\begin{lstlisting}{c}
Console.WriteLine("1 + 2 = {0}", Add(1, 2)};
\end{lstlisting}
And your source code should looks like the following:

\begin{lstlisting}{c}
using System;
using System.Runtime.InteropServices;
namespace ChapterTwo
{
	class Program
	{
		[DllImport("ChapTwo")]
		static extern int Add(int a, int b);
		static void Main(string[] args)
		{
		Console.WriteLine("1 + 2 = {0}", Add(1, 2));
		}
	}
}
\end{lstlisting}
\newpage
Finally, your program is ready to be executed. You can run:

''dotnet restore \&\& dotnet run''

And we have the following:

\includegraphics[width=\textwidth]{ChapTwoConsole}
It works as expected!
\newpage
\section{Some background}
There are few things happening when function with DllImport is called, if this is the first time the function is being called, the Runtime will immediately load the external library immediately and then load the symbol ''Add'' by default and then generate a P/Invoke stub for that function to support the call to external function.

The symbol is merely just that, a symbol that is exported by C Library and you can find a list of symbols by running ''objdump -T libChapTwo.so'' on your library and you'll have the following:

\includegraphics[width=\textwidth]{ChapTwoConsoleTwo}

You will notice that Add symbol is shown in the symbol table in your library, this is how CLR look up function by entry name.